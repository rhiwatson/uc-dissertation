\chapter{Conclusion}
\label{chap:conclusion}

In the preceding dissertation I have described the contributions that I have made in the field of particle astrophysics over the course of my graduate studies.
My primary focus was participation in the construction, operation, and analysis of the LZ dark matter direct detection experiment.
The projects which I worked on were varied, allowing me to build skills in a variety of sub fields.
Overall, my work revolved around the impact that electric fields have on the future of dark matter experiments.
The cathode high voltage feedthrough project directly determined the magnitude of the drift electric field within the LZ TPC.
Much of my simulations work involved the modelling and deployment of the spatially-varying electric fields within the xenon.
With regards to the accidental coincidence backgrounds, the broad conclusion is that the electric fields in the extraction region, particularly in the gas phase, are a significant driver of uncorrelated S1 and S2 pulses.
The Xenon Breakdown Apparatus collected a large amount of data which supported the theory of stressed area scaling for breakdown voltages, as well as evidence for activity preceding discharges.
While not tightly coupled to the electric fields, the multiple scatter / ultraheavy dark matter search in LZ is influenced by field nonuniformities and the isolated pulse rate, both of which are strong functions of the electric field.
Moving forward, both LZ and its potential successors will require accurate predictions for their electric fields. 
Intensive research and development is recommended in order to optimize for the effects observed over the course of this thesis.
