% (This file is included by thesis.tex; you do not latex it by itself.)

\begin{abstract}

Cold dark matter is a well motivated astrophysical phenomenon which is currently unexplained by the standard model of particle physics.
Presently, time projection chambers such as LUX-ZEPLIN are setting world leading limits in certain mass ranges, and possible improvements for next generation experiments are being researched.
Scaling up the size of experiments is one way to improve sensitivity, but poses certain challenges from a high voltage perspective.
In this dissertation I detail my  contributions to the LZ dark matter direct detection experiment's understanding of its electric fields and the impacts of high voltage on the future of TPCs in general.
In Chapter \ref{chap:chv} the testing and installation of LZ's novel cathode high voltage design are detailed, and in Chapters \ref{chap:fields},\ref{chap:sims}  the electric fields simulations for the LZ TPC are documented.
The sources of accidental coincidence backgrounds are examined in Chapter \ref{chap:accidental}, and results of a dielectric breakdown study of liquid xenon are reported in Chapter \ref{chap:xebra}.
Finally the results of an extension to the LZ dark matter search to be sensitive to Planck-scale masses are reported in Chapter \ref{chap:tracks}.
Throughout these work the electric field and grid voltages feature prominently, demonstrating their criticality to future generations of dark matter searches.

\end{abstract}
